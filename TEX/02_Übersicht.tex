\hypertarget{uxfcbersicht---energie--und-informationsfluss}{%
\section{Übersicht - Energie- und
Informationsfluss}\label{uxfcbersicht---energie--und-informationsfluss}}

\begin{quote}
{[}!TODO{]}+ ToDo - {[}x{]} Übersicht geben, wie Maschinenteile
zusammenhängen - {[}x{]} Was wirkt wie? - {[}x{]} Was bewirkt eine
Änderung dieser Größe bei anderen Größen? - {[}x{]} Was hängt wie
zusammen? - {[}x{]} evtl. formalisieren als Wirkungsgraph/Bondgraph etc.
- {[}x{]} Grundlage geben, aus der sich das Modelica Modell sozusagen
``von selbst'' ergibt --\textgreater{} im Modelica-Modell sollten vor
Allem Details und Simulationsspezifische Infos hinzugefügt werden. - {[}
{]} Diskretisierung der Regler --\textgreater{} Blockschaltbilder der
zeitdiskreten P, I und D-Regler - {[}x{]} Abtastung - {[}x{]}
Quantisierung auf 16bit signend Integer - {[}x{]} Reales D-GLied
--\textgreater{} Zeitkonstante für D-Regler
\end{quote}

\begin{quote}
{[}!question{]}- Soll ich die Details zur Abtastrate hier oder in
Parametrierung erwähnen? Ich denke, das ist besser unter Modellierung
aufgehoben. Da kann ich dann Modellierung und Parametrierung zusammen
behandeln
\end{quote}

Bevor das Simulationsmodell in Modelica implementiert werden soll, ist
es hilfreich zunächst einen Überblick über das Gesamtsystem
``rotierender Frequenzumformer'' zu erlangen und die Wirkungseinflüsse
der einzelnen Größen zu studieren.

\hypertarget{gesamtsystem}{%
\subsection{Gesamtsystem}\label{gesamtsystem}}

Wie schon in der Einführung gezeigt handelt es sich bei der betrachteten
Anlage im Wesentlichen um einen Motor-Generator-Verbund aus einem
Asynchronmotor und einem mit einer Synchron-Erregermaschine erregten
Synchrongenerator. Das System besteht also aus drei mechanisch
gekoppelten elektrischen Maschinen. Der Zusammenhang zwischen den
elektrischen Größen am Stator und am Rotor der elektrischen Maschinen
sowie die Kopplung mit der rotatorischen Bewegung wird über das
Induktionsgesetz durch die Spannung und den Fluss des magnetischen
Feldes charakterisiert.

{[}{[}Fluss\_Umformer.png{]}{]}

In Anlehnung an die Bondgraphen-Methode zeigt der
``Wort-Bondgraph''\footnote{Bezeichnung nach
  {[}@roddeckBondgraphenModellbildungUnd2019, p.10{]}.} (oder
``Wortgraph''\footnote{Bezeichnung nach
  {[}@roddeckBondgraphenModellbildungUnd2019, p.10{]}.}) XXX den
Energie- und Informationsfluss im Frequenzumformer.

Die Halbpfeile (XXX) repräsentieren den Energiefluss (die übertragene
Leistung) der zusammenhängenden \emph{Fluss-} (\emph{Flow}) und
\emph{Potential}größen (\emph{Effort}). Nach Konvention werden die
Potentialgrößen oberhalb und die Flussgrößen unterhalb der Halbpfeils
aufgetragen. Alternativ kann auch die Flussgröße direkt an einem Knoten
notiert werden, wenn die Größe für alle verbundenen Kanten konstant ist.
Im Gegensatz zu den formalen Bondgraphen können die Knoten des
Wortgraphen mehrere Größen und Verbindungen, die zum Aufstellen des
Zustandsraummodells nötig wären, unter einem Oberbegriff zusammenfassen
und vereinfachen (\emph{Objektorientierung}). Ist eine solche
detaillierte Darstellung gesucht, können die einzelnen Oberbegriffe
jeweils durch ihre korrespondierenden Bondgraph-Darstellungen ersetzt
werden. Bondgraphen für die einzelnen elektrischen Maschinen können
beispielsweise {[}@borutzkyBondGraphModelling2011, p.269ff.{]} entnommen
werden.

Ursache für den in XXX dargestellten Energiefluss ist der Vorgabe der
Spannung am Netz-Eingang. Sie bewirkt einen Stromfluss im Stator, der
einerseits zur Speicherung (Streuinduktivität) und Abgabe (omsche
Verluste) elektrischer Energie in den Statorwindungen führt und
andererseits zur Wandlung elektrischer Energie in magnetische führt.
Diese Energie wird über den Luftspalt (der wie ein Transformator das
Verhältnis der magnetischen Spannung zum Fluss wandelt\footnote{Beachte
  den Bezug auf Stator- (s) bzw. Rotorseite (s).}) auf den Rotor
übertragen. Im Rotor wird eine Spannung induziert, die einen Stromfluss
zur Folge hat. Über die Lorentzkraft folgt die Wandlung in kinetische
(rotatorische) Energie.

Während die Winkelgeschwindigkeit der Welle für angrenzenden Baugruppen
gleich ist, teilt sich das Motormoment auf. Ein Teil der Energie wird
über die Massenträgheit in der Welle gespeichert und ein Teil verlässt
das System durch Reibungseffekte (Lagerreibung, Luftwiderstand, \ldots).
Der größte Teil wird jedoch an den Synchro-Generator und die
Erregermaschine übertragen.

Die Hauptaufgabe der Erregermaschine ist es, die Spannung am Ausgang des
Reglersystems berührunglos auf den Rotor des Synchrogenerators zu
übertragen. Dies geschieht durch Wandlung der elektrischen Energie und
Übertragung über den Luftspalt der Maschine. Gekoppelt sind Generator
und Erregermaschine über den mitrotierenden Gleichrichter. Der
Synchro-Generator wandelt die eingebrachte mechanische Energie und die
Erregung über den Luftspalt in elektrische Energie.

Zusätzlich zu dem Energiefluss ist in XXX auch der Informationsfluss
angedeutet: Am Ausgang des Umformers wird der Spannungsabfall über der
Last gemessen und die Differenz zum Sollwert in den Regler
zurückgeführt. Der PID-Spannungsregler ist hier zusammengefasst
dargestellt. Ein ausführliches Blockschaltbild des Reglers zeigt XXX.
Aus der vom Regler berechneten Stellgröße wird über
Puls-Weiten-Modulation eines Steuertrafos die Statorspannung der
Erregermaschine erzeugt.

\hypertarget{pid-spannungsregler}{%
\subsection{PID-Spannungsregler}\label{pid-spannungsregler}}

Als Spannungsregler wird von Piller in dem Frequenzumformer ein
PID-Regler
(\textbf{P}roportional-\textbf{I}ntegral-\textbf{D}ifferential) mit
\emph{variabler Verstärkung} eingesetzt. Die variable Verstärkung dient
zum Erreichen eines besseren Einschwingverhaltens bei großen
Regelabweichungen
(\cite{ DigitalerSpannungsreglerSoftwaredokumentation }). XXX zeigt das
Blockschaltbild des Reglers. Der PID-Regler ist in Parallelstruktur
ausgeführt; an den Ausgängen der Regelglieder und am Gesamtausgang
werden die Stellgrößen begrenzt. Die einzelnen Regelglieder werden
zusätzlich zu möglichen Zeitkonstanten jeweils mit einem eigenen
Verstärkungsfaktor gewichtet. {[}{[}Blockschaltbild\_Regler.png{]}{]}

Implementiert ist der PID-Regler als digitaler Regler auf einem
Mikrocontroller. Dabei sind einige Punkte zu beachten: - Die ideale
Umsetzung eines Differenzierers (D-Glied) ist nicht möglich. Stattdessen
wird ein DT\_1-Glied verwendet, welches sich aus der Kombination eines
D-Glieds mit einem Verzögerungsglied 1.\textasciitilde Ordnung ergibt. -
Die Algorithmen für die Regelglieder arbeiten zeitdiskret. Beschrieben
werden diese Glieder mit Übertragungsfunktionen, die aus der
\(\mathcal{Z}\)-Transformation hervorgehen. - Die Erfassung der
Ausgangsspannung erfolgt nur zu bestimmten Zeitpunkten (Abtastung). Nach
\cite{ @DigitalerSpannungsreglerSoftwaredokumentation } erfasst der hier
betrachtete Spannungsregler 32 Messwerte in 5 Taktzyklen, wobei sich die
Taktfrequenz nach der Frequenz der Ausgangsspannung richtet. Die
Abtastrate für die Spannungsmessung beträgt also
  \begin{align}
  T_{\mathrm{mess}}^*&= \frac{5}{32}\cdot T_{\mathrm{Spannung}} = 0,000390625\,\mathrm s. \\
  \text{mit} \quad
  T_{\mathrm{Spannung}}&= \frac{1}{400\,\mathrm{Hz}}=0,0025\,\mathrm s
  \end{align}
  Weiterhin gibt \cite{ @DigitalerSpannungsreglerSoftwaredokumentation }
an, dass bei einer Spannungsfrequenz von \(f=400\,\mathrm{Hz}\) der
Regelalgorithmus nur alle zwei Messzyklen neu berechnet wird. Die
Periodendauer des zeitdiskreten Reglers beträgt also
  \begin{align}
  T_{mess}&=2\cdot T_{mess}^*= 0,00078125\,\mathrm s.
  \end{align} - Die Digitalisierung der Spannungsmessung bedeutet eine
Quantisierung der Werte. Die gemessenen Werte werden Reglerintern auf
\(2^{16}\) ganzzahlige vorzeichenbehaftete Werte (\emph{``16 bit signend
integer''}) abgebildet, also auf das Intervall \([-32768,32767]\). Um
den Zusammenhang zwischen der realen Spannung und der internen
Darstellung herzustellen, müssen Umrechnungsfaktoren angegeben werden.
