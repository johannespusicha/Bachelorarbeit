\chapter{Zusammenfassung und Ausblick}
\label{chap:ZusammenfassungAusblick}
%TODO: Zusammenfassung schreiben



%Notizen:
% Ergebnisse dieser Arbeit: In dieser Arbeit wurde
% - konnte ein objektorientiertes Modelle eines rotierenden Frequenzumformers hergelietet werden
% - Das Modell wurde mit Werten aus der Auslegung der Anlage parametriert. Wichtige Gleichungen zum Umformen der Auslegungsparameter wurden im Modell implementiert
% - Es wurden Eckpunkte zur Verifizierung & Validierung des Modells mittels Messergebnissen eines realen Umformers hergeleitet:
% 	- Das Gesamtmodell wird verifiziert, indem untersucht wird, ob das simulierte Verhalten den Erwartungen entspricht. Folgt die geregelte Spannung dem Sollwert, so kann geschlossen werden, dass der Regelkreis korrekt implementiert ist. Über die Untersuchung der Spannungsfrequenz am Ausgang der Anlage kann die Kopplung und Parametrierung der elektrischen Maschinen überprüft werden. Die Untersuchung der Leistungsflüsse im Modell und der Vergleich mit der Theorie stellt sicher, dass der Energieerhaltungssatz eingehalten wird und die Wirkrichtung der untersuchten Größen der Theorie folgt.
% 	- In der Validierung des Modells zeigt sich, ob die Anlage korrekt und hinreichend genau modelliert wurde. Über gemessene Kennlinien der elktrischen Maschinen kann die Parametrierung der Maschinen abgelglichen werden. 
% - Es wurden Parameterstudien zum Einfluss der Induktivitäten und Reglerparameter auf das dynamische Verhalten der Maschinen durchgeführt und das Potential zur Optimierung bestimmt. Es hat sich gezeigt, dass die Parametrierung der betrachteten Maschine nahe eines lokalen Minimums befindet. An dieser Stelle ist der Einfluss der beiden Parameterklassen ähnlich groß. Von den Induktivitäten beeinflussen hauptsächlich die Parameter der beiden Synchrongeneratoren das dynamische Verhalten. 
% - Aus den Erkenntnissen wurden Anwendungen für die Praxis der Auslegung abgeleitet. Dabei ist zu beachten, dass für eine Verallgemeinerung der Aussagen weitere Untersuchungen nötig sind. Im hier betrachteten Fall gilt, durch Variation der Maschinen- und Regelparameter das Dynamische Verhalten in ähnlich grßem Maß beeinflusst werden kann. Zur Optimierung der Anlage sind also beide Seiten zu betrachten, Aus praktischen Überlegungen ergibt sich jedoch eine Überlegenheit der Reglerparameter, da diese weniger Einschränkungen hinsichtlich Umsetzbarkeit und Kosten unterliegen.
% - Weiterhin können die Ergebnisse der Parameterstudie zur weiteren Verbesserung der Parametrierung der elektrischen Maschinen eingesetzt werden. Da teigt sich das größte Potential bei den beiden Hauptinduktivitäten der Synchrongeneratoren.
% 
% 