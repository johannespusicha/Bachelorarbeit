\chapter{Zusammenfassung und Ausblick}
\label{chap:ZusammenfassungAusblick}
In dieser Arbeit wurde die Erstellung eines Maschinenmodells zur Untersuchung der Schnittstelle der Auslegung und Regelung elektrischer Maschinen am Beispiel eines rotierenden Frequenzumformers behandelt. Verwendet wurde dazu eine objektorientierte Modellierungsmethode mit der Möglichkeit, Modelle unbestimmter Wirkrichtung zu erstellen. Gewählt wurde diese Methode, um ein hierarchisch strukturiertes, vielseitig an- und wiederverwendbares Modell zu erhalten. Das Modell wurde mit Werten aus der Auslegung einer Beispielanlage parametriert. Wichtige Gleichungen zum Umformen der Auslegungsparameter wurden dazu im Modell implementiert. Bei Bedarf wurden für einige Objekte Ersatzmodelle (z.B. zur Frequenzmessung) verwendet, die numerisch effizienter zu lösen sind.

Zu dem umgesetzten Modell wurden Eckpunkte zur Verifizierung und Validierung des Modells an Messergebnissen der Beispielanlage vorgestellt. Zur Verifikation des Gesamtmodells wurden drei charakteristische Ergebnisse der Simulation untersucht:
\begin{itemize}
	\item Die Betrachtung der Leistungsflüsse im Modell gibt Aufschluss über die Einhaltung des Energieerhaltungssatz. Werden weiterhin die Verlustleistungen untersucht kann festgestellt werden, ob die Leistungsflussrichtung und damit die Wirkrichtung den Modellvorgaben entsprechen.
	\item Über den Vergleich der geregelten Ausgangsspannung mit der Sollwertrampe des Reglers kann der Regelkreis überprüft werden. Folgt die Ausgangsspannung dem Sollwert, so kann die Implementierung des Regelkreis als korrekt angesehen werden.
	\item Mit der Untersuchung der Spannungsfrequenz am Ausgang des Umformers wird die Kopplung und Parametrierung der elektrischen Maschinen überprüft. Stimmen die aus der Winkelgeschwindigkeit berechnete Frequenz und die aus dem Spannungssystem bestimmte Frequenz überein, ist die Funktion der Kopplung gewährleistet. Kann weiterhin beobachtet werden, dass die Frequenz auch nach einem Lastsprung noch im gewünschten Bereich liegt, so kann gefolgert werden, dass die Parametrierung des Asynchronmotors plausibel ist.
\end{itemize}
In der Validierung des Modells zeigt sich, ob die Anlage korrekt und hinreichend genau modelliert wurde. Über gemessene Kennlinien der elektrischen Maschinen konnte die Parametrierung der Maschinen mit den Messergebnissen abgeglichen werden.

An dem verifizierten und validierten Modell wurde eine Parameterstudie über die Reglerparameter und die Induktivitäten der elektrischen Maschinen durchgeführt. Ausgewertet wurde die Parameterstudie unter zwei Sichtweisen:\begin{itemize}
	\item Es wurde der Einfluss der Induktivitäten und Reglerparameter auf das dynamische Verhalten der Maschinen untersucht und das Potential zur Optimierung bestimmt. Es hat sich gezeigt, dass die Parametrierung der betrachteten Maschine sich nahe eines lokalen Minimums befindet. An dieser Stelle ist der Einfluss der beiden Parameterklassen ähnlich groß. Von den Induktivitäten beeinflussen hauptsächlich die Parameter der beiden Synchrongeneratoren das dynamische Verhalten. Aus diesen Erkenntnissen wurden Anwendungen für die Praxis der Auslegung abgeleitet. Dabei ist zu beachten, dass für eine Verallgemeinerung der Aussagen weitere Untersuchungen nötig sind. Im hier betrachteten Fall gilt, dass durch Variation der Maschinen- und Regelparameter das dynamische Verhalten in ähnlich großem Maß beeinflusst werden kann. Zur Optimierung der Anlage sind also beide Domänen (Maschinen- und Reglerparameter) zu betrachten, aus praktischen Überlegungen ergibt sich jedoch eine Überlegenheit der Reglerparameter, da diese weniger Einschränkungen hinsichtlich Umsetzbarkeit und Kosten unterliegen.
	\item Andererseits können die Ergebnisse der Parameterstudie zur weiteren Verbesserung der Parametrierung der elektrischen Maschinen eingesetzt werden. Da zeigt sich das größte Potential bei den beiden Hauptinduktivitäten der Synchrongeneratoren.
\end{itemize}

Nach Betrachtung dieser beiden Anwendungen des Modells bleibt ein großes Feld an Anwendungsmöglichkeiten des Modells offen. So könnte das Modell im besonderen zur automatisierten Einstellung der Reglerparameter (\emph{Autotuning}) oder zur Optimierung des Wirkungsgrads zum Einsatz kommen. Gerade der zweite Punkt bietet ein lohnenswertes Ziel, wenn berücksichtigt wird, dass der Frequenzumformer für die Verwendung im Dauerbetrieb konzipiert ist. Neben diesen weiteren Anwendungsfeldern des Modells bleibt jedoch auch Raum zur Verbesserung des Modells und der Parametrierung, Berücksichtigung weiterer Details, wie der verschiedenen Verlustleistungen, der Temperaturabhängigkeit und der Wärmeleitung, die zur Wirkungsgradoptimierung untersucht werden. Ein weiterer Schritt könnte auch die Reduktion der Modellordnung (ROM) durch Neuronale Netze und maschinelles Lernen sein, die zwar nicht dieselbe Einsicht in die Modellordnung bieten, jedoch Effizienzvorteile in großen Parameterstudien aufweisen können. Ebenso ist auch die Verwendung des Modells zur regelungstechnischen Systemidentifikation denkbar, die beispielsweise eine Optimierung des Reglers zum Ziel haben könnte. Auch kann nicht abschließend geklärt werden, in wie weit die hier gewonnen Erkenntnisse auf weitere Anlagen und Modelle übertragen werden können. An dieser Stelle kann diesen Fragestellungen nicht weiter nachgegangen werden; sie bleiben offen für weiterführende Untersuchungen.
