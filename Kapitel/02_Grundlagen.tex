\chapter{Grundlagen}
\label{chap:Grundlagen}

%--------------------------------------------------------
\section{Aus dem Bereich der Regelungstechnik}
\label{sec:GrundlagenRegelungstechnik}
Aus dem Bereich der Regelungstechnik werden zur Modellierung und Simulation einige Grundlagen zum PID-Regler und zur Behandlung zeitdiskreter Regler benötigt, die im Folgenden beschrieben werden.

\paragraph{PID-Regler}
\label{par:PID-Regler}
Die Spannungsregelung des Frequenzumformers geschieht mit einem PID-Regler (\textbf{P}ro\-por\-ti\-o\-nal-\textbf{I}ntegral-\textbf{D}ifferential-Regler), einem Standardregler, der in vielen Bereichen eingesetzt wird. Für den \emph{idealen PID-Regler} lassen sich die Übertragungsfunktion im Laplace-Bereich und Sprungantwort im Zeitbereich durch Addition der drei Regelglieder aufstellen:
\begin{align}
    K_{\mathrm{PID}}(s) &= K_{\mathrm{P}} + \frac{K_{\mathrm{I}}}{s} + K_{\mathrm{D}}s \label{eq:IdealerPID} \\
    h_{\mathrm{PID}}(t) &= (K_{\mathrm{P}} + K_{\mathrm{I}}t)\sigma(t) + K_{\mathrm{D}}\delta(t) \label{eq:IdealerPIDSprungantwort}
\end{align}
Der ideale PID-Regler ist in der Realität nicht umsetzbar, da der Zählergrad des Reglerpolynoms größer als der Nennergrad ist. Nach \cref{eq:IdealerPIDSprungantwort} würde die Sprungantwort des Reglers einen unendlich hohen $\delta(t)$-Impuls enthalten, der technisch nicht realisierbar ist. Daher wird für den \emph{realen PID-Regler} das D-Glied mit einem DT-Glied angenähert. Die Übertragungsfunktion und die Sprungantwort des realen PID-Reglers ergeben sich dann zu\begin{align}
    K_{\mathrm{PIDT}}(s) &= K_{\mathrm{P}} + \frac{K_{\mathrm{I}}}{s} + K_{\mathrm{D}}\frac{sT}{1+sT} \label{eq:realerPID} \\
    \intertext{und}
    h_{\mathrm{PIDT}}(t) &= (K_{\mathrm{P}} + K_{\mathrm{I}}t)\sigma(t) + K_{\mathrm{D}} e^{-\frac t T}.
\end{align}
\paragraph{Zeitdiskreter Regler}\label{par:ZeitdiskreterRegler}
Als \emph{Digitaler Regler} auf einem Mikrocontroller wird der Regler als Algorithmus auf abgetastete Messsignale angewandt. Dazu muss eine zeitdiskrete Form des Reglers angegeben werden. Es werden abgetastete Signale der Form \begin{equation}
    x^*(t) = \sum_{k=0}^\infty x(kT_{\mathrm{s}})\delta(t-kT_{\mathrm{s}})
\end{equation} mit der Abtastzeit $T_{\mathrm{s}}$ betrachtet. Der Dirac-Impuls dient dazu, die Messwerte an den jeweiligen Abtastzeitpunkt zu verschieben. Wird nun die Laplace-Transformierte von $x^*(t)$ berechnet 
\begin{equation}
    X^*(s) = \mathcal{L}\{x^*(t)\} = \sum_{k=0}^\infty x(kT_{\mathrm{s}})e^{-skT_{\mathrm{s}}} 
\end{equation}
und der Ausdruck $e^{sT_{\mathrm{s}}}$ durch die neue Variable $z$ ersetzt ergibt sich eine Transformation zur Untersuchung von Zahlenfolgen $x[k]$, die $\mathcal{Z}$-Transformation
\begin{equation}
    X(z) = \mathcal{Z}\{x[k]\} = \sum_{k=0}^\infty x[k]z^{-k}.
\end{equation} Wichtige Eigenschaften und Korrespondenztabellen zur $\mathcal{Z}$-Transformation sind unter Anderem in \cite{mbihiTableZtransforms2018} und \cite[S.~112-114]{unbehauenRegelungstechnikZustandsregelungenDigitale2009} angegeben.

Zur Überführung kontinuierlicher Regler mit Hilfe der $\mathcal{Z}$-Transformation auf in die zeitdiskrete Darstellung muss noch der Einfluss der Abtastung der Messsignale berücksichtigt werden. Vereinfacht kann angenommen werden, dass die Abtastung nach der Methode \emph{Sample and Hold}, das heißt durch Abgreifen eines Messwertes und Halten des Messwertes bis zum nächsten Tastzyklus. Beschrieben wird dieser Vorgang mit einem \emph{Zero-Order-Hold}-Glied (ZOH), das dem Regler vorgeschaltet wird. \cref{eq:ZOH_Zeit,eq:ZOH_Laplace} geben die Darstellung des ZOH im Zeit- und im Laplacebereich an.
\begin{align}
    h(t) &= \sigma(t)*\left(\delta(t) - \delta(t-T_{\mathrm{s}})\right) \label{eq:ZOH_Zeit}\\
    H(s) &= \frac{1 - e^{-sT_{\mathrm{s}}}}{s} \label{eq:ZOH_Laplace}
\end{align}
Für einen Regler im $\mathcal{Z}$-Bereich ergibt sich damit
\begin{equation}
\begin{split}\label{eq:ReglerZBereich}
    K(z) &= \mathcal{Z}\left\{ \left.\mathcal{L}^{-1}\left\{ H(s)K(s) \right\}\right|_{kT_{\mathrm{s}}} \right\} \\
    &= \mathcal{Z}\left\{ \left.\mathcal{L}^{-1}\left\{ (1 - e^{-sT_{\mathrm{s}}})\frac{G(s)}{s} \right\}\right|_{kT_{\mathrm{s}}} \right\} \\
    &= \frac{z-1}{z} \mathcal{Z}\left\{ \mathcal{L}^{-1}\left.\left\{\frac{G(s)}{s}\right\}\right|_{kT_{\mathrm{s}}} \right\}. 
\end{split}
\end{equation}

%--------------------------------------------------------------
\section{Aus dem Bereich elektrischer Maschinen}
\label{sec:GrundlagenEMaschinen}
Der im Rahmen dieser Arbeit untersuchte Frequenzumformer verwendet ein Motor-Generator-Set zur Wandlung. Dabei kommen zwei Arten elektrischer Maschinen zum Einsatz: Zum Einen eine Asynchronmaschine mit Kurzschlussläufer als Motor und eine Schenkelpolsynchromaschine mit bürstenloser Erregung als Generator, ausgeführt als Innenpolmaschine. In diesem Abschnitt sollen die für die Modellbildung benötigten Grundlagen dieser Maschinen eingeführt werden. Da hier nicht alle Aspekte im Detail behandelt werden können, sei auch auf die Literatur verwiesen, die hier Verwendung gefunden hat (vor Allem \cite{binderElektrischeMaschinenUnd2012}, \cite{beckElektrischeEnergietechnikEinfuhrung2008} und \cite{mullerGrundlagenElektrischerMaschinen2005}).

Beiden Maschinenarten gemeinsam ist die Verwendung eines (bezogen auf den Stator) sich drehenden magnetischen Felds (Drehfeldmaschinen) zur elektro-mechanischen Kopplung. Dabei erfolgt die elektro-magnetische Wandlung in beiden Fällen durch kreisförmig angeordnete Spulen, die im Blechpaket des Stators eingelassen sind. Die Erzeugung dieses magnetischen Feldes erfolgt bei beiden Maschinen elektrisch (\emph{elektrische Erregung}).

Charakterisierendes Unterscheidungsmerkmal beider Maschinen ist das Verhältnis der mechanischen Drehzahl zur Drehzahl des magnetischen Feldes: Während der Rotor der Synchronmaschine dem Magnetfeld in einem konstanten Winkelabstand folgt (\(-\frac{\pi}{2} < \varphi < \frac{\pi}{2}\)), tritt bei der Asynchronmaschine Schlupf zwischen der mechanischen Drehzahl und der Magnetfelddrehzahl auf. Dieser wird mit \(s=\frac{n_{\mathrm{syn}} - n}{n_{\mathrm{syn}}}\) (\(n_{\mathrm{syn}}\) -- Synchrondrehzahl des Magnetfelds) bezeichnet.

\subsection{Asynchronmaschine}
Charakterisierend für die Asynchronmaschine sind das elektrische Ersatzschaltbild und die Drehzahl-Drehmoment-Kennlinie. Beide beschreiben das stationäre Betriebsverhalten der Maschine und sind grundlegend zur Modellierung, Parametrierung und Verifizierung der Asynchronmaschine.
\paragraph{Elektrisches Ersatzschaltbild}
Das elektrische Ersatzschaltbild für jeden der drei Stränge der Asynchronmaschine im stationären Zustand zeigt \cref{fig:ESB_ASM}. Die Asynchronmaschine wird als Drehtransformator betrachtet mit dem Stator als Primär- und dem Läufer als Sekundärseite. Die Statorwicklungen werden durch die Ständerhauptinduktivität $L_{\mathrm{h}}$, der Ständerstreuinduktivität $L_{\mathrm{s\sigma}}$ und einem ohmschen Widerstand $R_{\mathrm{s}}$ der Wicklungen wiedergegeben. Der Kurzschlussläufer wird als kurzgeschlossene Wicklung dargestellt. Durch Umrechnung der Rotorgrößen auf eine dreisträngige Wicklung ergibt sich die Sekundärseite des Drehtransformators mit der Rotorstreuinduktivität $L'_{\mathrm{r\sigma}}$ und dem Schlupfabhängigen Widerstand $\nicefrac{R'_{\mathrm{r}}}{s}$
\label{subsec:Ersatzschaltbilder}
\begin{figure}[h]
    \centering
    \def\THIS{\tikztostart}
\def\normalcoord(#1){coordinate(#1)}
\def\showcoord(#1){coordinate(#1) node[circle, red, draw, inner sep=1pt, pin={[red, overlay, inner sep=0.5pt, font=\tiny, pin distance=0.1cm, pin edge={red, overlay}]45:#1}](){}}
\let\coord=\normalcoord
%\let\coord=\showcoord

\begin{circuitikz}
    \node [ocirc] (Plus) at (0,0) {};
    \node [ocirc] (Minus) at (0,-3.5) {};
    \draw (Plus) to[short, i>^=$\underline{I}_{\mathrm{s}}$] ++(1,0) to[R, R=$R_{\mathrm{s}}$] ++(2,0) to[L, L=$L_{\mathrm{s\sigma}}$] ++(2,0) node [circ](Hmp){} |- ++(-.5,-.5) to[L, l_=$L_{\mathrm{m}}$] ++(0,-2) -- ++(.5,0) to[short, i=$\underline{I}_m$] (Hmp|- Minus) node[circ](Hmm){} -- (Minus);
    
    \draw (Hmp) |- ++(.5,-.5) to[R, R=$R_{\mathrm{v}}$] ++(0,-2) -| (Hmm);
    \draw (Hmp) to[L, L=$L_{\mathrm{r\sigma}}$] ++(2,0) to[R, R=$\nicefrac{R^\prime_{\mathrm{r}}}{s}$] ++(2,0) to[short, i<=$\underline{I}^\prime_{\mathrm{r}}$]  (\THIS |- Minus) -- (Hmm);
    \draw (Plus) to[open, v=$\underline{U}_{\mathrm{s}}$] (Minus);
    
    \path (Plus.center) \coord(Plus);
    \path  (Minus.center) \coord(Minus);
    \path (Hmp.center) \coord(Hmp);
    \path (Hmm.center) \coord(Hmm);
\end{circuitikz}
    \caption{T-Ersatzschaltbild eines Strangs der Asynchronmaschine mit Kurzschlussläufer}
    \label{fig:ESB_ASM}
\end{figure}

\paragraph{Drehzahl-Drehmoment-Kennlinie}
Die Drehzahl-Drehmoment-Kennlinie der Asynchronmaschine zeigt \cref{fig:KennlinieASM}. Die Kennlinie ergibt sich aus der Leistungsbilanz der Maschine mit den Strom-Spannungs-Beziehungen aus dem Ersatzschaltbild zu \begin{equation}
    M_{\mathrm{i}}=m_{\mathrm{s}} \frac{p}{\omega_{\mathrm{s}}} U_{\mathrm{s}}^{2} \frac{s(1-\sigma) X_{\mathrm{s}} X_{\mathrm{r}}^{\prime} R_{\mathrm{r}}^{\prime}}{\left(R_{\mathrm{s}} R_{\mathrm{r}}^{\prime}-s \sigma X_{\mathrm{s}} X_{\mathrm{r}}^{\prime}\right)^{2}+\left(s R_{\mathrm{s}} X_{\mathrm{r}}^{\prime}+X_{\mathrm{s}} R_{\mathrm{r}}^{\prime}\right)^{2}}
\end{equation} mit der Strangzahl $m_{\mathrm{s}}$, der Polpaarzahl $p$, der Synchronwinkelgeschwindigkeit $\omega_{\mathrm{s}}$, dem Streufaktor $\sigma$ und dem Schlupf $s$.

Charakteristisch für die Asynchronmaschine sind das ausgeprägte Maximum (\emph{Kippmoment}) und Minimum des Momentenverlaufs und der von den beiden Extremstellen begrenzte näherungsweise lineare Arbeitsbereich der Kennlinie. Für $0 < n < n_{\mathrm{syn}}$ befindet sich die Maschine im Motorbetrieb, ab $n > n_{\mathrm{syn}}$ wechselt sie in den Generatorbetrieb.

\begin{figure}
    \centering
    \begin{tikzpicture}[scale=0.8]
\tikzmath{
    \p = 1;
    \ASMms = 3;
    \Us = 100;
    \ASMsigma = 0.067;
    \Xs = 3;
    \Xr = \Xs;
    \Rr = 0.008*\Xr;
    \Rs = 0.01*\Xs;
    \ASMfs = 50;
    \nsyn = 60/2/pi*\ASMfs/\p / (600/12);
    function drehzahl(\s) {
        return (1 - \s)*\ASMfs*60/\p/2/pi;
    };
    \nk = drehzahl(\Rr/\ASMsigma/\Xr) / (600/12);
    \nn = \nk + 0.8*(\nsyn-\nk);
    function schlupf(\n) {
        return 1 - \n*\p/\ASMfs*2*pi/60;  
    };
    function drehmoment(\n) {
        return \ASMms * \p/(2*pi*\ASMfs) * \Us^2 * schlupf(\n)*(1-\ASMsigma)*\Xs*\Xr*\Rr / ((\Rs*\Rr - schlupf(\n)*\ASMsigma*\Xs*\Xr)^2 + (schlupf(\n)*\Rs*\Xr + \Xs*\Rr)^2);
    };
    \Mk = drehmoment(\nk*(600/12))/(250/3.2);
    \Mn = drehmoment(\nn*(600/12))/(250/3.2);
}

    \draw[->, thick, >=triangle 45] (-.5, 0) -- (12, 0) node[right] {$n,\, \omega$};
    \draw[->, thick, >=triangle 45] (0, -3.2) -- (0, 3.2) node[above] {$M_{\mathrm{i}}$};
    
    \node[below right, yshift=-4pt] (nsyn) at (\nsyn, 0) {$n_{\mathrm{syn}}$};
    \node[below, yshift=-4pt] (nk) at (\nk, 0) {$n_{\mathrm{K}}$};
    \node[below, yshift=-4pt] (nn) at (\nn, 0) {$n_{\mathrm{N}}$};
    \node[above] (Mk) at (\nk, \Mk) {$M_{\mathrm{K}}$};
    \node[above right] (Mn) at (\nn, \Mn) {$M_{\mathrm{Nenn}}$};
 
    \draw[thick] (nsyn) -- (\nsyn, 0);
    \draw[thick] (nk) -- (\nk, 0);
    \draw[thick] (nn) -- (\nn, 0);
    \draw[thick] (Mk) -- (\nk, \Mk);
    \draw[thick] (Mn) -- (\nn, \Mn);
    \draw[|->, thick, >=triangle 45] (\nsyn,0) ++(0,-1) node[below, yshift=-6pt] {$0$} -- ++(-1,0) node[left] {$s$}; 
    \draw[mark=*, fill=white, mark options={thick}, only marks, mark size=3pt] plot coordinates {
        (\nk, \Mk) %Kippmoment
        (\nn, \Mn) %Nennmoment
    };
    
    \draw[domain=-.5:12, smooth, variable=\n, very thick, samples=50] plot ({\n}, {drehmoment(\n*(600/12))/(250/3.2)});
\end{tikzpicture}
    \caption{Drehmoment der Asynchronmaschine in Abhängigkeit der Drehzahl $n$. }
    \label{fig:KennlinieASM}
\end{figure}

\subsection{Synchronmaschine}
\label{sec:Synchronmaschine}
Zur Beschreibung der Synchronmaschine sollen im Folgenden das elektrische Ersatzschaltbild der Maschine und die beiden heute gebräuchlichsten Systeme zur elektrischen Erregung betrachtet werden.
\paragraph{Elektrisches Ersatzschaltbild}
Das elektrische Ersatzschaltbild des Schenkelpol-Syn\-chron\-ge\-ne\-ra\-tors zeigt \cref{fig:ESB_SM}. Im Statorkreis der Maschine (siehe \cref{fig:ESB_SM_Stator}) wird durch das Läuferdrehfeld eine ein symmetrisches Drehstromsystem mit der Spannung $\underline{U}_{\mathrm{p}}$ induziert. Wird weiterhin der Statorkreis durch einen Verbraucher geschlossen, so fließt ein Strom $\underline{I}_{\mathrm{s}}$, der neben Verlusten ein Gegenfeld aufbaut, beschrieben durch die fiktive Induktivität $L_{\mathrm{m}}$, welche mit derselben magnetischen Flussänderung $\frac{\mathrm{d} \Phi}{\mathrm{dt}}$ verbunden ist wie die reale Läuferwicklung. Da im Statorkreis nur sinusförmige periodische Zeitverläufe auftreten können diese als komplexe Zeiger dargestellt werden, für den Rotorkreis (siehe \cref{fig:ESB_SM_Rotor}) gilt dies im Allgemeinen nicht.

Aufgrund der Schenkelpole der Maschine ist der Luftspalt nicht konstant und weist eine Achsigkeit auf. Das dreiphasige System des Stators lässt sich als komplexer Raumzeiger darstellen. Als Bezugssystem des Raumzeigers wird ein rotorfestes mitrotierendes Koordinatensystem mit den Achsen $d$ und $q$ so gewählt, dass die $d$-Achse im räumlichen Maximum des Raumzeigers und die $q$-Achse orthogonal dazu liegt. Die beiden Komponenten des Raumzeigers können als Größen einer zweipoligen Ersatzmaschine interpretiert werden, deren Pole in der $d$-Achse und deren Pollücken in der $q$-Achse liegen. Die Angabe der magnetischen Größen des Stators und der elektrischen Größen des Rotors geschieht in Bezug auf dieses Koordinatensystem. Diese Transformation heißt auch \emph{Park-Transformation} oder \emph{d,q-Transformation}.
\begin{figure}
    \centering
    \begin{subfigure}[c]{.66\textwidth}
        \def\THIS{\tikztostart}
\def\normalcoord(#1){coordinate(#1)}
\def\showcoord(#1){coordinate(#1) node[circle, red, draw, inner sep=1pt, pin={[red, overlay, inner sep=0.5pt, font=\tiny, pin distance=0.1cm, pin edge={red, overlay}]45:#1}](){}}
\let\coord=\normalcoord
%\let\coord=\showcoord

\begin{circuitikz}
    \node[ocirc] (Plus) at (0,0) {};
    \node[ocirc] (Minus) at (0,-3) {};
    
    \draw (Plus) to[short, i>^=$\underline{I}_{\mathrm{s}}$] ++(1,0) to[R, R=$R_{\mathrm{s}}$] ++(2,0) to[L, L=$L_{\mathrm{s\sigma}}$] ++(2,0) to[L, L=$L_{\mathrm{md/q}}$] ++(2,0) to[sV, v=$\underline{U}_{\mathrm{p}}$] (\THIS |- Minus) -- (Minus);
    \draw (Plus) to[open, v=$\underline{U}_{\mathrm{s}}$] (Minus);
\end{circuitikz}
        \subcaption{Stator\label{fig:ESB_SM_Stator}}
    \end{subfigure}
    \hskip -3em
    \begin{subfigure}[c]{.33\textwidth}
        \def\THIS{\tikztostart}
\def\normalcoord(#1){coordinate(#1)}
\def\showcoord(#1){coordinate(#1) node[circle, red, draw, inner sep=1pt, pin={[red, overlay, inner sep=0.5pt, font=\tiny, pin distance=0.1cm, pin edge={red, overlay}]45:#1}](){}}
\let\coord=\normalcoord
%\let\coord=\showcoord

\begin{circuitikz}
    \draw (0,0) node[] (Lmr) {} to[L, L=$L_{\mathrm{md/q}}$] ++(2,0) to[short, i<=$i_{\mathrm{r}}(t)$] ++(1,0) node[ocirc] (RPlus) {} to[open, v=$u_{\mathrm{e}}(t)$] ++(0,-2) node[ocirc] (RMinus) {} to[R, R=$R_{\mathrm{rd/q}}$] (Lmr |- RMinus) to[L, L=$L_{\mathrm{r\sigma d/q}}$] (Lmr.center);
    
\end{circuitikz}
        \subcaption{Rotor\label{fig:ESB_SM_Rotor}}
    \end{subfigure}
    \caption{Ersatzschaltbild der Synchronmaschine für $d$- und $q$-Achse des Läufers}
    \label{fig:ESB_SM}
\end{figure}

\paragraph{Erregersystem}
Zur Übertragung der Erregerleistung auf den Rotor des Synchrongenerators werden in der Praxis vor Allem zwei Verfahren eingesetzt:\begin{itemize}
    \item Eine Möglichkeit zur Erregung ist der Einsatz von Schleifringen und Kohlebürsten, die über den Gleitkontakt eine elektrische Verbindung mit einer regelbaren Gleichstromquelle herstellen. Bezeichnet wird diese Methode als \emph{statische Erregung}, da zur Erregung keine rotierenden Maschinen eingesetzt werden. Nachteilig ist, dass der Einsatz der Kohlebürsten und Schleifringe mit Verschleiß und Wartungsarbeiten verbunden ist und unter gewissen Bedingungen es durch den Abrieb der Kohlebürsten zu Bürstenfeuer kommen kann.
    \begin{figure}
        \centering
        \def\THIS{\tikztostart}
\def\normalcoord(#1){coordinate(#1)}
\def\showcoord(#1){coordinate(#1) node[circle, red, draw, inner sep=1pt, pin={[red, overlay, inner sep=0.5pt, font=\tiny, pin distance=0.1cm, pin edge={red, overlay}]45:#1}](){}}
\let\coord=\normalcoord
%\let\coord=\showcoord

\begin{circuitikz}[scale=.5,/tikz/circuitikz/bipoles/length=1cm]
    % Hauptknoten
    \node [circ] (HStar) at (0,0) {};
    \node [circ] (EStar) at (12,-6) {};
    % Stern Haupt
    \draw (HStar) to[L] ++(90:3) node[ocirc] (Ha) {};
    \draw (HStar) to[L] ++(-30:3)  node[ocirc] (Hb) {};
    \draw (HStar) to[L] ++(-150:3) node[ocirc] (Hc) {};
    % Erregung
    \node (Err) at (EStar |- HStar) {};
    \draw (Err) ++(1,1.5) node[ocirc] {} to[short] ++(-1,0) to[L] ++(0,-3) to[short] ++(1,0) node[ocirc] {};
    % Stern Erreger
    \draw (EStar) to[L] ++(60:3) node[] (Ea) {};
    \draw (EStar) to[L] ++(-60:3)  node[] (Eb) {};
    \draw (EStar) to[L] ++(180:3) node[] (Ec) {};
    % Kabel
    \draw (Ea.center) -- (Ec |- Ea) -- ++(0,-2.2) -- ++(-4.5,0) node[circ] (L1) {};
    \draw (Ec.center) -- ++(-3,0) node[circ] (L2) {};
    \draw (Eb.center) -- (Ec |- Eb) -- ++(0,2.2) -- ++(-1.5,0) node[circ] (L3) {};
    % Dioden
    \draw (L1 |- L2) to[D] (L1 |- Ea);
    \draw (L2) to[D] (L2 |- Ea);
    \draw (L3 |- L2) to[D] (L3 |- Ea);
    \draw (L1 |- Eb) to[D] (L1 |- L2);
    \draw (L2 |- Eb) to[D] (L2);
    \draw (L3 |- Eb) to[D] (L3 |- L2);
    % Kabel zu Hauptrotor
    \draw (L3 |- Ea) to[short] (L2 |- Ea) node[circ]{} to[short] (L1 |- Ea) node[circ]{} to[short]  (HStar |- Ea) to[L] (HStar |- Eb) to[short] (L1 |- Eb) node[circ]{} to[short] (L2 |- Eb) node[circ]{} to[short] (L3 |- Eb);
    % Beschriftung
    \node [] (Slabel) at ($(HStar) + (16,0)$) {Ständer};
    \node [] (Rlabel) at (Slabel |- EStar) {Rotor};
    \node [] (Hlabel) at ($(HStar) + (0,-10)$) {Hauptgenerator};
    \node [] (Plabel) at (L2 |- Hlabel) {Diodenrad};
    \node [] (Elabel) at (EStar |- Hlabel) {Erregermaschine};
    % Trennlinien
    \coordinate (hbar) at ($(Hc)!.5!(Ea)$);
    \draw [dashed] (Hc |- hbar) +(-1,0) -- (Slabel |- hbar);
    \coordinate (vbar1) at ($(Hb)!.5!(L1 |- Ea)$);
    \draw [dashed] (vbar1 |- Ha) -- (vbar1 |- Hlabel);
    \coordinate (vbar2) at ($(L3)!.5!(Ec)$);
    \draw [dashed] (vbar2 |- Ha) -- (vbar2 |- Hlabel);
\end{circuitikz}
        \caption{Bürstenloses Erregersystem, eigene Darstellung nach \cite[S. 557]{mullerGrundlagenElektrischerMaschinen2005}}
        \label{fig:burstenloseErregung}
    \end{figure}
    \item Daher findet häufig auch die \emph{bürstenlose Erregung} Verwendung. Sie umgeht den Gleitkontakt durch den Einsatz eines weiteren kleineren Synchrongenerators ausgeführt als \emph{Außenpolmaschine}. Mit dem so erzeugten Drehstrom wird die Erregerwicklung des Hauptgenerators über eine mitrotierende Diodenbrücke gespeist. 
    
    Nachteil dieser Methode ist, dass die Dynamik der Regelstrecke durch die \emph{Erregermaschine} verlangsamt wird und eine Entregung („negative Erregung“) nicht möglich ist. Die Struktur dieses Erregersystems zeigt \cref{fig:burstenloseErregung}. 
\end{itemize}

%----------------------------------------------------
\section{Aus dem Bereich der Modellierung und Simulation}
\label{sec:GrundlagenModellierung}

\subsection{Arten der Modellierung}
\label{subsec:ArtenModellierung}
%% White-/Black-/Gray-Box Modellierung
%% Signalfluss - Objektorientierte Modellierung

\subsection{Modellierung elektrischer Maschinen}
\label{subsec:ModellierungElektrischerMaschinen}

\subsection{Übersicht über Simulationsprogramme}
\label{subsec:Simulationsprogramme}
