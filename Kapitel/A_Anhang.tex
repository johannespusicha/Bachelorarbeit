\chapter{Anhang}
\label{chap:Anhang}
%\drawlist
\section{Übersicht über Parameter der Dynamischen Messung mit Reglerveränderung}
\label{sec:Reglerparameter}

\begin{longtable}[]{ll}
    \caption{Veränderter D-Anteil (Standardwert: $\mathrm{UgenCtrlD_G}=27648$)}
    \label{tab:Parameter-D-Messung}
    \tabularnewline
    \toprule
    Versuch     & $\mathrm{UgenCtrlD_G}$ \\
    \midrule
    \endhead
    9.1.1.1.1/2 & 26112        \\
    9.1.1.2.1/2 & 26880        \\
    9.1.1.3.1/2 & 28160        \\
    9.1.1.4.2 & 29184        \\
    9.1.4.6.1/2 & 256          \\
    9.1.4.6.3/5 & 5376         \\
    9.1.4.6.4/6 & 13824        \\
    9.1.4.6.7/8 & 20736        \\
    \bottomrule
\end{longtable}

\begin{longtable}[]{ll}
    \caption{Veränderter I-Anteil (Standardwert: $\mathrm{UgenCtrlI_G}=304$))}
    \label{tab:Parameter-I-Messung}
    \tabularnewline
    \toprule
    Versuch     & $\mathrm{UgenCtrlI_G}$ \\
    \midrule
    \endhead
        9.1.2.1.1/2 & 0            \\
        9.1.2.2.1/2 & 152          \\
        9.1.2.3.1/2 & 456          \\
        9.1.2.4.1/2 & 608          \\
        9.1.2.6.1/2 & 912          \\
    \bottomrule
\end{longtable}

\begin{longtable}[]{lll}
    \caption{Veränderter konstanter P-Anteil, eingestellt mit Begrenzung des PP-Glieds,\\Standardwerte: $\mathrm{UgenCtrlPP_G}=2048$, $\mathrm{UgenCtrlPP_{UL}}=8192$, $\mathrm{UgenCtrlPP_{LL}}=6144$)}
    \label{tab:Parameter-P-Messung}
    \tabularnewline
    \toprule
    Versuch     & $\mathrm{UgenCtrlPP_{LL}}$ & $\mathrm{UgenCtrlPP_{UL}}$ \\
    \midrule
    \endhead
        9.1.3.1.1/2 & 3072         & 3072  \\
        9.1.3.2.1/2 & 4608         & 4608  \\
        9.1.3.3.1/2 & 8192         & 8192  \\
        9.1.3.4.1/2 & 12288        & 12288 \\
    \bottomrule
\end{longtable}

\begin{longtable}[]{lll}
    \caption{PI-Regler mit veränderten konstanten Verstärkungen, eingestellt mit Begrenzung des PP-Glieds,\\Standardwerte: siehe \cref{tab:Parameter-P-Messung},\\D-Glied ausgeschaltet: $\mathrm{UgenCtrlD_G=0}$}
    \label{tab:Parameter-PI-Messung}
    \tabularnewline
    \toprule
    Versuch     & $\mathrm{UgenCtrlPP_{LL}}$ & $\mathrm{UgenCtrlPP_{UL}}$ \\
    \midrule
    \endhead
        9.1.3.1.1/2 & 3072         & 3072  \\
        9.1.3.2.1/2 & 4608         & 4608  \\
        9.1.3.3.1/2 & 6144         & 6144  \\
        9.1.3.4.1/2 & 12288        & 12288 \\
    \bottomrule
\end{longtable}
%TODO: Entkommentieren
%\begin{landscape}
%\begin{figure}
%    \centering
%    \oldincludegraphics[]{Bilder/MessungReglerSweep.pdf}
%    \caption{Zeitverläufe der dynamischen Messung mit Reglerveränderung}
%    \label{fig:MessungReglerSweep}
%\end{figure}
%\end{landscape}
%TODO: Tabelle mit statischen Messergebnissen
%TODO: Spannungen aus dynamischer Messung ohne Reglerveränderung
\section{Zeitverläufe der Spannungen aus den Parameterstudien}
\begin{figure}[h!]
    \centering
    \begin{subfigure}{\linewidth}
        \centering
        \oldincludegraphics[width=\textheight-2.5in, angle=90]{Bilder/ParameterSweep.pdf}
        \subcaption{Hauptinduktivitäten, Statorstreuinduktivitäten}
        \label{fig:InduktivitatenSweepA}
    \end{subfigure}
    \caption{Zeitverläufe der simulierten Spannungen aus der Parameterstudie der Induktivitäten. Die Legende rechts gibt die Farbzuordnung zu dem ausgewerteten Punkt. Aufgetragen sind 10er-Schritte aus der Parameterstudie}
\end{figure}

\begin{figure}[ht]\ContinuedFloat
    \begin{subfigure}{\textwidth}
        \centering
        \includegraphics{Bilder/ParameterSweep_RotorStreuinduk.pdf}
        \subcaption{Rotorstreuinduktivitäten des Synchrongenerators. Aufgetragen sind 10er-Schritte aus der Parameterstudie.}
        \label{fig:InduktivitatenSweepB}
    \end{subfigure}
    \begin{subfigure}{\textwidth}
        \centering
        \includegraphics{Bilder/simulation_reglerSweepSpannungen.pdf}
        \subcaption{Reglerparameter. Die Legende rechts gibt den jeweiligen Variationsfaktor einer Kurve}
    \label{fig:SpannungenReglerSweep}
    \end{subfigure}
    \caption{Zeitverläufe der simulierten Spannungen aus der Parameterstudie.}
    \label{fig:ZeitverlauefeSpannungenParametersweep}
\end{figure}
