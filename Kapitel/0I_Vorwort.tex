\chapter{Vorwort}
\label{sec:Vorwort}
Die vorliegende Arbeit beschäftigt sich mit der Anwendung einer modellbasierten Entwicklungsmethodik im Bereich des Entwurfs und der Auslegung elektrischer Maschinen und ihrer Regler. Mit dem Ziel, das Optimierungspotential einer elektrischen Anlage zu bestimmen, wird ein Vorgehen zur Modellierung, Verifizierung, Validierung und Auswertung des Modells und Interpretation der Ergebnisse aufgezeigt. Zur Modellierung wird eine objektorientierte Methodik verwendet, die es ermöglicht, Teile des Modells oder das Gesamtmodell in anderen Anwendungsfällen wiederzuverwenden. 

Durchgeführt wurde dieser Prozess am Beispiel eines rotierenden Frequenzumformers. Mit einer Parameterstudie wurde das Modell ausgewertet, um zu untersuchen, wie groß der Einfluss der Maschinen- und Reglerparameter auf das dynamische Verhalten des Systems ist. Es hat sich gezeigt, dass sich die dazu gestellte Zielfunktion im untersuchten Parameterbereich konvex verhält. Eine Optimierung der Parameter in diesem Bereich führt also stets zu einem globalen Minimum. Weiterhin konnte aus den Ergebnissen dieser Studie abgeleitet werden, dass im aktuellen Entwicklungsstadium des untersuchten Umformers der Einfluss der Maschinen- und Reglerparameter auf das dynamische Verhalten ähnlich groß ist. Unter Berücksichtigung der mit der Realisierung der Parameter verbundenen Kosten können daraus betriebswirtschaftliche Optimierungsstrategien abgeleitet werden.

Entstanden ist diese Arbeit im Rahmen meiner Tätigkeit als Werksstudent bei der Firma \textsc{Piller Power Systems} mit Sitz in Osterode am Harz. Besonderer Dank gilt an dieser Stelle Herrn Dr.-Ing Peter Methfessel für die beständige Unterstützung. Die vielen hilfreichen Anregungen haben einen entscheidenden Beitrag zu dieser Arbeit geleistet. Für die Betreuung seitens der Universität sei Herrn Prof. Dr.-Ing Christian Bohn gedankt. Ebenso sei allen Mitarbeitern der Firma Piller gedankt, die benötigte Informationen weitergegeben und an der Aufnahme von Messungen mitgewirkt haben. Namentlich seien hier erwähnt (in alphabetischer Reihenfolge) Herr Gunther Brandt, Herr Arno Hausmann, Herr Kai Killig, Herr Dr.-Ing Omar Okla, Herr Karsten Rott, Herr Rüdiger Scherff. 