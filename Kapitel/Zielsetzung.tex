\hypertarget{zielsetzung}{%
\section{Zielsetzung}\label{zielsetzung}}

\hypertarget{titel}{%
\subsection{Titel:}\label{titel}}

Erstellung eines Maschinenmodells zur Validierung der Auslegung und
Regelung elektrischer Maschinen - Objektorientierte Modellierung eines
rotierenden Frequenzumformers

\hypertarget{Ziel}{%
\subsection{Ziel:}\label{ziel}}

Die Regelung elektrischer Maschinen geschieht auch heute noch häufig mit
dem Einsatz von PI- bzw. PID-Reglerstrukturen. Das Einstellen der
Reglerparameter wird dabei, sofern kein selbsteinstellender Regler
(Autotuning) verwendet wird, manuell und mit Hilfe von Erfahrungswerten
oder Einstellregeln durchgeführt. Dazu muss jedoch zuerst die zu
regelnde Maschine existieren; die Anpassung des Reglers kann also erst
nach Entwicklung der Anlage, mit der Produktion eines ersten Prototypen
erfolgen. Um diesen Entwicklungszyklus zu verkürzen, soll im Rahmen
dieser Arbeit am Beispiel eines rotierenden Frequenzumformers ein Modell
der elektrischen Maschine aufgestellt werden, an dem der Regler
eingestellt und erprobt werden kann.

In dieser Arbeit soll zunächst das Gesamtsystem ``rotierender
Frequenzumformer'' hinsichtlich Informations- und Energiefluss zu
dokumentiert werden, um einen Überblick über das System zu geben. Im
Anschluss soll ein Modell des Systems aufgestellt, simuliert und zur
Charakterisierung des Systems eingesetzt werden. Dazu ist es notwendig,
geeignete Anforderungen an das Modell (Gütekriterien, Einschränkungen,
Anwendungsbereiche) zu stellen. Weiterhin muss das aufgestellte Modell
mit Parametern aus der Auslegung der Maschine versehen und mittels
Messungen an einer realen Maschine validiert werden. Der Aufbau des
Modells soll dabei akausal (d.h. ohne a priori Vorgabe der
Energieflussrichtung) und objektorientiert (d.h. unterteilt in
physikalisch abgegrenzte Baugruppen) erfolgen, um eine möglichst große
Wiederverwendbarkeit der Modelle zu ermöglichen.

Die zentralen Fragen, die in dieser Arbeit beantwortet werden sollen,
sind daher:

\begin{itemize}
\tightlist
\item
  Wie muss ein Modell aufgebaut werden, um hinreichend genau die reale
  Maschine abzubilden?
\item
  Wenn es unterschiedliche Modellvarianten gibt, wie unterscheiden Sie
  sich und welche Vor- und Nachteile haben sie?
\item
  Welche Randbedingungen sind zwingend einzuhalten und welche
  Randbedingungen sind optional?
\item
  Welchen Einfluss haben die Maschinenparameter und die Regelparameter
  auf das dynamische Verhalten der Maschine?
\item
  Optional: Welchen Einfluss hat eine Variation der Eingangsgrößen
  (Maschinen- und Regelparameter) sowie die Maschinengröße auf das
  dynamische Verhalten der Maschine?
\end{itemize}

\hypertarget{zeitlicher-ablauf}{%
\subsection{Zeitlicher Ablauf:}\label{zeitlicher-ablauf}}

