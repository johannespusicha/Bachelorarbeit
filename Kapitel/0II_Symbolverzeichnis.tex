\chapter{Symbolverzeichnis}
\begin{longtable}[l]{Ll}
\parbox{2cm}{\section*{Regelungstechnik}} \\
K, k & Verstärkungsfaktor \\
T & Periodendauer, Zeitkonstante \\
T_{\mathrm{s}} & Abtastrate \\
\delta(t) & Dirac-Delta-Distribution \\
\sigma(t) & Sprungfunktion \\
\mathcal{L}\{...\} & Laplace-Transformation \\
\mathcal{Z}\{...\} & Z-Transformation \\
(f*g)(x) & Faltung der Funktionen $f$ und $g$\\

\parbox{2cm}{\section*{Elektromagnetik}} \\
f & Frequenz \\
G & elektrischer Leitwert \\
I,\,i(t) & elektrischer Strom \\
L & Induktivität \\
m & Strangzahl \\
N & Windungszahl \\
N_{\mathrm{eff.}} & effektive Windungszahl \\
P & Wirkleistung \\
\mathrm{Pf} & Leistungfaktor \\
p & Polpaarzahl \\
Q & Nutzahl \\
q & Lochzahl \\
R & elektrischer Widerstand \\
S & Scheinleistung \\
s & Schlupf \\
U,\,u(t) & elektrische Spannung \\
V_{\mathrm{m}} & magnetische Spannung \\
X,x & Reaktanz, bezogene \\
y_{\mathrm{Q}} & Nutschritt \\
Z & Impedanz \\

\Delta\gamma_{\mathrm{c}} & Spulenweite \\
\sigma & Streuziffer \\
\Phi & magnetischer Fluss \\
\varphi & Phasenwinkel \\
\xi_{\mathrm{c}} & Sehnungsfaktor \\
\xi_{\mathrm{z}} & Zonenfaktor \\

\parbox{2cm}{\section*{Mechanik}} \\
J & Trägheitsmoment \\
M & Drehmoment \\
n & Drehzahl \\
\omega & Winkelgeschwindigkeit \\

\parbox{2cm}{\section*{Signalverarbeitung}} \\
\mathrm{ISE} & integrierter quadratischer Fehler \\
\mathrm{MAE} & mittlerer absoluter Fehler \\
\mathrm{mean}(x) & Mittelwertsfunktion \\

\parbox{2cm}{\section*{Indizes}} \\
(\_)\,_{\mathrm{P}} & Proportional-Glied \\
(\_)\,_{\mathrm{I}} & Integral-Glied \\
(\_)\,_{\mathrm{D}} & Differential-Glied \\
(\_)\,_{\mathrm{v}} & Verstärkungsfaktor \\
(\_)\,_{\mathrm{d}} & Dämpfungsfaktor \\
(\_)\,_{\mathrm{m}} & haupt- \\
(\_)\,_{\mathrm{\sigma}} & streu- \\
(\_)\,_{\mathrm{s}} & Bezogen auf Ständerseite \\
(\_)\,_{\mathrm{r}} & Bezogen auf Rotorseite \\
(\_)\,_{\mathrm{d}} & Bezogen auf rotorfeste d-Achse \\
(\_)\,_{\mathrm{q}} & Bezogen auf rotorfeste q-Achse \\
(\_)\,_{\mathrm{AC}} & Wechselgröße \\
(\_)\,_{\mathrm{DC}} & Gleichgröße \\
(\_)\,^{\mathrm{'}} & transiente Größe \\
(\_)\,^{\mathrm{''}} & subtransiente Größe \\
\end{longtable}
